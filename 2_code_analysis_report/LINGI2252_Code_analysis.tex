% LINGI2252 - Measures & Maintenance
% Code Analysis
\documentclass[11pt, a4paper]{article}
\usepackage[utf8]{inputenc}
\usepackage[UKenglish]{babel}
\usepackage{graphicx}				% Use pdf, png, jpg, or eps§ with pdflatex; use eps in DVI mode
\usepackage{xcolor}
\usepackage{listings}
\usepackage{hyperref}
\usepackage{array}
\usepackage{longtable}
\usepackage{multirow}
\usepackage[babel=true]{csquotes}


\usepackage[T1]{fontenc}

\lstset{%
	basicstyle=\ttfamily\footnotesize,
	commentstyle=\color{green!90!black},
	frame=single,
	keywordstyle=\bfseries\color{blue},
	language=python,
	numberstyle=\color{gray},
%	tabsize=2,
}


\hypersetup{%
	colorlinks=true,
	linkcolor=blue,
	urlcolor=blue
}


\newcommand{\tbf}[1]{\textbf{#1}}
\newcommand{\tit}[1]{\textit{#1}}

\newcommand{\pyl}{\textsf{Pylint}}


\def\blurb{\textsc{Université catholique de Louvain\\
  École polytechnique de Louvain\\
  Pôle d'ingénierie informatique}}
\def\clap#1{\hbox to 0pt{\hss #1\hss}}%
\def\ligne#1{%
  \hbox to \hsize{%
    \vbox{\centering #1}}}%
\def\haut#1#2#3{%
  \hbox to \hsize{%
    \rlap{\vtop{\raggedright #1}}%
    \hss
    \clap{\vbox{\vfill\centering #2\vfill}}%
    \hss
    \llap{\vtop{\raggedleft #3}}}}%
\begin{document}

\begin{titlepage}
\thispagestyle{empty}\vbox to 1\vsize{%
  \vss
  \vbox to 1\vsize{%
    \haut{\raisebox{-2mm}{\includegraphics[width=2.5cm]{logo_epl.jpg}}}{\blurb}{\raisebox{-5mm}{\includegraphics[scale=0.20]{ingi_logo.png}}}
    \vfill
    \ligne{\Huge \textbf{\textsc{LINGI2252}}}
     \vspace{5mm}
    \ligne{\huge \textbf{\textsc{Software Engineering : measures and maintenance}}}
     \vspace{15mm}
    \ligne{\Large \textbf{\textsc{Assignment step 2: \\Assessing program quality}}}
    \vspace{5mm}
    \ligne{\large{\textsc{November 3, 2014}}}
    \vfill
    \vspace{5mm}
    \ligne{%
         \textsc{Professor\\Kim Mens}
      }
      \vspace{10mm}
    }%
    \ligne{%
         \textsc{Group 23\\Michael Heraly\\Thibault Gerondal}
      }
      \vspace{5mm}
  \vss
  }
\end{titlepage}



\newpage
\section*{Presentation of Reddit}

Reddit is a popular social networking service that allows its members to share content as text posts or links.
Members can vote ``up'' or ``down'' on a content to promote it to other members.
The content is organized in subcategories, called \tit{subreddits}.
Reddit is an open-source project, and its code is available on Github\footnote{\url{https://github.com/reddit/reddit}}.


\section{Methodology}

\subsection*{Reddit \& \pyl{}}

There was already a \texttt{pylintrc} file in the project.
This file disables some errors that can be generated by \pyl{}.
We decided to keep their \texttt{pylintrc} file because it filters the output that is already consequent.

\smallskip
More precisely, here are the errors that Reddit disabled for \pyl{}: \\
\\
\begin{tabular}{|c|m{10cm}|}
\hline
\textbf{\small{Error code}} & \textbf{\small{Signification}} \\
\hline
\hline
E1103 	& X has no Y member (but some types could not be inferred) \\
\hline
W0212 	& Access to a protected member of X class \\
\hline
W0223	& Method Y is abstract in class X but not overridden \\
\hline
C0103	& Invalid name ``\%s'' (should match \%s) \\
\hline
C0111	& Missing docstring \\
\hline
W0142	& Used * or ** magic \\
\hline
R0201	& Method could be a function \\
\hline
R0915	& Too many statements \\
\hline
I0011	& Locally disabling \dots (don't need to see things we've explicitly disabled) \\
\hline
\end{tabular}

\medskip
Particularly, we found that disabling the C0103 error is not a good idea.
Respecting naming conventions is a good practice, but it appears that Reddit doesn't really care about that.


\bigskip
\subsection*{Analysis}
After installing Reddit and its dependencies\footnote{\url{https://github.com/reddit/reddit/wiki/Install-guide}}, we have run \pyl{} on the source code.
We used the \tit{parseable} output feature that \pyl{} offers, and created a script to create a CVS file format from the parseable file.
This way, we created a spreadsheet containing the informations about errors/warnings/refactors/conventions/fatal detected by \pyl{}: the file concerned, at which line, code of the error, and the message associated. This was a practical help to analyse the \pyl{} output, and compute the statistics.


\newpage
\section{Type of errors}

\begin{longtable}{|l|c|r|m{9cm}|}
\hline
\textbf{\small{Error}} & \textbf{\small{Occ\footnote{Number of occurences}}} & \textbf{\small{RFP\footnote{Rate of false positive(s)}}} & \textbf{\small{Message}} \\
\hline
\hline
E1101 & 225 & 17.78\% & An object (variable, function, \dots) is accessed for a non-existent member. \\
\hline
W0201 & 144 & 0.00\% & Attribute X defined outside \_\_init\_\_ \\
\hline
W0613 & 98 & 0.00\% & Unused argument X \\
\hline
W0612 & 91 & 8.79\% & Unused variable X \\
\hline
R0913 & 91 & 8.79\% & Too many arguments X \\
\hline
W0621 & 88 & 9.09\% & Redefining name X from outer scope \\
\hline
E0611 & 49 & 0.00\% & No name X in module Y \\
\hline
R0914 & 49 & 4.08\% & Too many local variables \\
\hline
W0622 & 40 & 2.50\% & Redefining built-in X \\
\hline
W0231 & 38 & 2.63\% & \_\_init\_\_ method from base class X is not called \\
\hline
C0324 & 34 & 0.00\% & Comma not followed by a space \\
\hline
W0311 & 27 & 0.00\% & Bad indentation \\
\hline
C0321 & 26 & 3.85\% & More than one statement on a single line \\
\hline
R0902 & 24 & 4.17\% & Too many instance attributes \\
\hline
W0221 & 23 & 0.00\% & Arguments number differs from X method \\
\hline
W0403 & 22 & 0.00\% & Relative import X, should be Y \\
\hline
E0602 & 21 & 80.95\% & Undefined variable X \\
\hline
W0232 & 21 & 19.05\% & Class has no \_\_init\_\_ method \\
\hline
W0102 & 20 & 0.00\% & Dangerous default value X as argument \\
\hline
R0911 & 12 & 0.00\% & Too many return statements \\
\hline
W0404 & 10 & 0.00\% & Reimport X \\
\hline
R0904 & 9 & 22.22\% & Too many public methods \\
\hline
W0233 & 8 & 0.00\% & \_\_init\_\_ method from a non direct base class X is called \\
\hline
W0702 & 8 & 0.00\% & No exception type(s) specified \\
\hline
E0211 & 7 & 100.00\% & Method has no argument \\
\hline
C0203 & 7 & 0.00\% & Metaclass method X should have mcs as first argument \\
\hline
E1123 & 6 & 83.33\% & Passing unexpected keyword argument X in function call \\
\hline
C0322 & 6 & 0.00\% & Operator not preceded by a space \\
\hline
C0202 & 5 & 0.00\% & Class method X should have cls as first argument \\
\hline
W0105 & 5 & 0.00\% & String statement has no effect \\
\hline
R0912 & 5 & 0.00\% & Too many branches \\
\hline
E1120 & 4 & 0.00\% & No value passed for parameter X in function call \\
\hline
E0213 & 4 & 0.00\% & Method should have \enquote{self} as first argument \\
\hline
W0631 & 4 & 50.00\% & Using possibly undefined loop variable X \\
\hline
E0710 & 3 & 0.00\% & Raising a new style class which doesn't inherit from BaseException \\
\hline
W0703 & 3 & 0.00\% & Catching too general exception X \\
\hline
F0401 & 2 & 50.00\% & Unable to import X \\
\hline
W0106 & 2 & 0.00\% & Expression X is assigned to nothing \\
\hline
R0901 & 2 & 0.00\% & Too many ancestors \\
\hline
W0602 & 2 & 100.00\% & Using global for X but no assigment is done \\
\hline
E1002 & 1 & 100.00\% & Use of super on an old style class \\
\hline
W0122 & 1 & 100.00\% & Use of exec \\
\hline
E0702 & 1 & 100.00\% & Raising X while only classes, instances or string are allowed \\
\hline
W0101 & 1 & 0.00\% & Unreachable code \\
\hline
E0711 & 1 & 0.00\% & NotImplemented raised - should raise NotImplementedError \\
\hline
W0701 & 1 & 0.00\% & Raising a string exception \\
\hline
E0102 & 1 & 0.00\% & X already defined line Y \\
\hline
E0101 & 1 & 0.00\% & Explicit return in \_\_init\_\_ \\
\hline
E0203 & 1 & 100.00\% & Access to member X before its definition line Y \\
\hline
\caption{\label{messages} Analyzed messages}
\end{longtable}


%\newpage
\bigskip
\section{Statistics}

\subsection*{Precision}

We checked the source code to determine if these errors were true or false-positives.
We computed the precision on all the messages given by \pyl{} (as seen in Table~\ref{messages}). This way, we can have a good idea of the precision of \pyl{}.\\

We have computed the precision for each category of the informations given by \pyl{} :

\begin{longtable}{|l|r|r|m{9.2cm}|}
\cline{2-3}
\multicolumn{1}{c|}{} & \textbf{\small{Precision}} & \textbf{\small{Messages analyzed}} \\
\cline{2-3}
\hline
Precision on Warning & 95.89\% & 657 \\
Precision on Error & 77.84\% & 325 \\
Precision on Refactoring & 93.22\% & 192 \\
Precision on Convention & 98.71\% & 78 \\
Precision on Fatal & 50.00\% & 2 \\
\hline
Total Precision & 90.90\% & 1254 \\
\hline
\end{longtable}

As we can see, \pyl{} is really good at detecting conventions, refactoring and warning issues and is a little bit less accurate for advices about errors. \\

Fatal happens if an error occurred which prevented \pyl{} from doing further processing. We got two \enquote{unable to import X} in our case because these modules are not mandatory to run Reddit.

\subsection*{Recall}

As the project is too big to be analysed in details, we have selected 5 random files to compute the \tit{recall} statistic. \\

We didn't manage to find much more than what Pylint did.
In \texttt{r2/lib/base.py}, the following errors were not reported :

\begin{lstlisting}[caption= 3 useless functions]
def try_pagecache(self):
    pass
def pre(self): pass
def post(self): pass
\end{lstlisting}

\medskip
\pyl{} could generate three errors for these functions but didn't.


Another file analysed is \texttt{r2/lib/authorize/interaction.py} :

\begin{lstlisting}[caption= test data in the file..]
# useful test data:
test_card = dict(AMEX       = ("370000000000002"  , 1234),
                 DISCOVER   = ("6011000000000012" , 123),
                 MASTERCARD = ("5424000000000015" , 123),
                 VISA       = ("4007000000027"    , 123),
                 # visa card which generates error codes based
                 # on the amount
                 ERRORCARD  = ("4222222222222"    , 123))

test_card = Storage((k, CreditCard(cardNumber=x,
                                   expirationDate="2011-11",
                                   cardCode=y)) for k, (x, y) in
                    test_card.iteritems())

test_address = Address(firstName="John",
                       lastName="Doe",
                       address="123 Fake St.",
                       city="Anytown",
                       state="MN",
                       zip="12346")
\end{lstlisting}

These data are not used in the code at all.
It looks like a developer tested the code with these data, but did not remove them.

We haven't found any remarks to tell in the three others files. The recall we calculated is 0.82\%.


\section{Bad things}

\subsection{3 nested for-loops}

We found 3 nested for-loops, 2 times in a row, in a single function.
This is a good example of the kind of \tit{bad smell} we can encounter in the Reddit project.
There should be a refactoring of this function, and it would certainly make the code cleaner to decompose this function in some smaller functions.
\pyl{} did not give a hint of this \tit{bad smell}.

\begin{lstlisting}[caption=\texttt{r2/lib/inventory.py} at lines 261-292]
for campaign in all_campaigns:
  camp_dates = set(get_date_range(campaign.start_date, \
					campaign.end_date))
  sr_names = tuple(sorted(campaign.target.subreddit_names))
  daily_impressions = campaign.impressions / campaign.ndays

  for location in locations:
    if location and not location.contains(campaign.location):
      # campaign's location is less specific than location
      continue

    for date in camp_dates.intersection(dates):
      booked_dict[date][location][sr_names] += daily_impressions

  # calculate inventory for each target and location on each date
  datekey = lambda dt: dt.strftime('%m/%d/%Y') if datestr else dt

ret = {}
for target in targets:
  name = make_target_name(target)
  subreddit_names = target.subreddit_names
  ret[name] = {}
  for date in dates:
    pageviews_by_location = {}
    for location in locations:
      # calculate available impressions for each location
      booked_by_target = booked_dict[date][location]
      pageviews_by_sr_name = pageviews_dict[location]
      pageviews_by_location[location] = get_maximized_pageviews( \
          subreddit_names, booked_by_target, pageviews_by_sr_name)
    # available pageviews is the minimum from all locations
    min_pageviews = min(pageviews_by_location.values())
    ret[name][datekey(date)] = max(0, min_pageviews)
\end{lstlisting}

\newpage 
\subsection{Hard to read flow structure}

The following code is really hard to read. First, it uses a really unusual control flow structure, the for-else structure. The Python documentation tells this : \enquote{Loop statements may have an else clause; it is executed when the loop terminates through exhaustion of the list (with for) or when the condition becomes false (with while), but not when the loop is terminated by a break statement}. The second trick, to understand this code, is about the scope of variables in Python. The scope of a variable does not exist in \texttt{for}, \texttt{while} and \texttt{if} structures. So the scope of any variable is delimited by the scope of the function. So, bin still exists after the for-else structure. 

\begin{lstlisting}[caption=\texttt{r2/lib/nymph.py} at lines 123-135]
bins = []

for image in small_images:
  # find a bin to fit in
  for bin in bins:
    if bin.has_space_for(image):
      break
  else:
    # or give up and create a new bin
    bin = SpriteBin((0, sprite_height, sprite_width,\
    		 sprite_height + image.height))
    sprite_height += image.height + SPRITE_PADDING
    bins.append(bin)
  bin.add_image(image)
\end{lstlisting}


\pyl{} sends us the following error for this part of the code \enquote{Using possibly undefined loop variable bin}. This was a hard one.

\section{Crazy if's}

\pyl{} warned us about a case of ``too many branches'', and it was right.
In the \texttt{r2/lib/pages/pages.py}, from line 460 to 661, there were 48 \lstinline|if| \dots \lstinline|else|, in a single function.
This shows that there is clearly a need to refactor and simplify this code, because adding comments would not make the code really more comprehensible.



\subsection{Bomb}

This is the most useless code found in this project. This code is never called and is really useless, as you can read. But it's funny. We don't really know why this part of code is there.

\begin{lstlisting}[caption=\texttt{r2/lib/utils/utils.py} at lines 1200-1215]
class Hell(object):
    def __str__(self):
        return "boom!"

class Bomb(object):
    @classmethod
    def __getattr__(cls, key):
        raise Hell()

    @classmethod
    def __setattr__(cls, key, val):
        raise Hell()

    @classmethod
    def __repr__(cls):
        raise Hell()
\end{lstlisting}



\end{document}
