% LINGI2252 - Measures & Maintenance
% Code Analysis
\documentclass[12pt, a4paper]{article}
\usepackage[utf8]{inputenc}
\usepackage[UKenglish]{babel}
\usepackage{graphicx}				% Use pdf, png, jpg, or eps§ with pdflatex; use eps in DVI mode
\usepackage{xcolor}
\usepackage{listings}
\usepackage{hyperref}
\usepackage{array}
\usepackage{longtable}
\usepackage{multirow}
\usepackage[babel=true]{csquotes}

\usepackage[T1]{fontenc}

\lstset{%
	basicstyle=\ttfamily\small,
	commentstyle=\color{green!90!black},
	frame=single,
	keywordstyle=\bfseries\color{blue},
	language=python,
	numberstyle=\color{gray},
%	tabsize=2,
}


\hypersetup{%
	colorlinks=true,
	linkcolor=blue,
	urlcolor=blue
}


\newcommand{\tbf}[1]{\textbf{#1}}
\newcommand{\tit}[1]{\textit{#1}}
\newcommand{\ttt}[1]{\texttt{#1}}

\newcommand{\pyl}{\textsf{Pylint}}


\def\blurb{\textsc{Université catholique de Louvain\\
  École polytechnique de Louvain\\
  Pôle d'ingénierie informatique}}
\def\clap#1{\hbox to 0pt{\hss #1\hss}}%
\def\ligne#1{%
  \hbox to \hsize{%
    \vbox{\centering #1}}}%
\def\haut#1#2#3{%
  \hbox to \hsize{%
    \rlap{\vtop{\raggedright #1}}%
    \hss
    \clap{\vbox{\vfill\centering #2\vfill}}%
    \hss
    \llap{\vtop{\raggedleft #3}}}}%
\begin{document}

\begin{titlepage}
\thispagestyle{empty}\vbox to 1\vsize{%
  \vss
  \vbox to 1\vsize{%
    \haut{\raisebox{-2mm}{\includegraphics[width=2.5cm]{logo_epl.jpg}}}{\blurb}{\raisebox{-5mm}{\includegraphics[scale=0.20]{ingi_logo.png}}}
    \vfill
    \ligne{\Huge \textbf{\textsc{LINGI2252}}}
     \vspace{5mm}
    \ligne{\huge \textbf{\textsc{Software Engineering : measures and maintenance}}}
     \vspace{15mm}
    \ligne{\Large \textbf{\textsc{Assignment step 3: \\Providing critiques on the code-checking tool}}}
    \vspace{5mm}
    \ligne{\large{\textsc{November 14, 2014}}}
    \vfill
    \vspace{5mm}
    \ligne{%
         \textsc{Professor\\Kim Mens}
      }
      \vspace{10mm}
    }%
    \ligne{%
         \textsc{Group 23\\Michael Heraly\\Thibault Gerondal}
      }
      \vspace{5mm}
  \vss
  }
\end{titlepage}



\newpage
~
\bigskip

\section*{Presentation of \pyl{}}

\pyl{} is a free and open-source code-checking tool written in Python.
It has a lot of capabilities : Coding Standard, Error detection, Refactoring help, Fully customizable, Editor integration, IDE integration, UML diagrams and Continuous integration.
In our case, we will only analyse the suggestions given by \pyl{}, namely : Coding Standard, Error detection and Refactoring help.

\bigskip
\section*{Analysis}
\smallskip
\subsection*{Methodology}

We used our previous analysis of Reddit to have a concrete basis to analyse the messages generated by \pyl{}.
To do so, we have to take into account the fact that Reddit has already a \texttt{pylintrc} file in the project.
This file disables some errors that can be generated by \pyl{}.
We decided to modify this \texttt{pylintrc} file to see how relevant is the tool out of the box.

\medskip
Our first step was to review the disabled messages.
We have reactived them one at a time to see if they had a big impact.
Then, we examined all messages given by \pyl{} and focused on the ones that could be improved.
When possible, we checked several files to see if the error was applied the same way everywhere.
Since the Reddit project is quite huge and not really clean, we think that our analysis is quite relevant.
\newpage
\subsection*{Disabled messages}
Here are the \pyl{} codes that Reddit disabled for \pyl{}. 
We suspect that these messages generate a lot of noise without real reason. \\

\begin{longtable}{|l|r|m{9cm}|}
\hline
\textbf{\small{Code}} & \textbf{\small{Diff\footnote{Number of messages added}}} & \textbf{\small{Message}} \\
\hline
\hline
E1103 & 212 & X has no Y member (but some types could not be inferred) \\
\hline
W0212 & 2240  & Access to a protected member of X class \\
\hline
W0223 & 123 & Method Y is abstract in class X but not overridden \\
\hline
C0103 & 33663 & Invalid name ``\%s'' (should match \%s) \\
\hline
C0111 & 3363 & Missing docstring \\
\hline
W0142 & 87 & Used * or ** magic \\
\hline
R0201 & 333 & Method could be a function \\
\hline
R0915 & 31 & Too many statements \\
\hline
\caption{\label{messages} The disabled messages by Reddit can generate a lot of noise. }
\end{longtable}


As we can see in Table~\ref{messages}, some enabled codes (e.g. C0111) generate a lot of noise. 
Let's analyse the utility of some of these messages.
\bigskip
\subsection*{Noise}
\subsubsection*{C0111 : Missing docstring}

As we can see, this code generates 3363 messages.
In general, writing a docstring is really useful to help other developers to read the code and understand what the function does.
But sometimes, adding a docstring is not necessary because the function name already gives a good hint of what it does.


\medskip
\subsubsection*{C0103 : Invalid name ``\%s''  (should match \%s)}

This code generates 33663 messages.
And we can easily understand why\dots \\
This code is about naming convention.
For example, a variable named ``u'' in the code will be reported by \pyl{} with the code C0103 (Invalid variable name ``u'').
But this code also applies to methods, class, constants, functions, modules and attributes names.

%\medskip
\subsubsection*{W0212 : Access to a protected member of X class}

Used when a protected member (i.e. class member with a name beginning with an underscore) is accessed outside the class or a descendant of the class where it has been defined.
This convention is not always respected in Reddit as we can see that \pyl{} gives 2240 messages.


\medskip
\subsubsection*{E1103 : X has no Y member (but some types could not be inferred)}

Used when a variable is accessed for a nonexistent member, but astng (Python Abstract Syntax Tree New Generation) was not able to interpret all possible types of this variable.
This error is often triggered but it is nearly always a false-positive since astng was not able to inferred the type of the variable.


\medskip
\subsubsection*{E1101 : An object (variable, function, \dots) is accessed for a non-existent member.}

Used when an object (variable, function, \dots{}) is accessed for a non-existent member.
This message is often a false-positive since the object members can have been created dynamically.
But dynamically creating an object member is quite a bad practice.
It is noise since the reason of this situation is that an object member has been declared outside of the \lstinline|__init__| method which is detected by \pyl{} as W0201 : Attribute X defined outside \lstinline|__init__|.

\medskip
\subsubsection*{W0613 : Unused argument X}
Used when an argument is not used in the body of its function or method.
This easily happens if the method has been overriden.
And this is noise since the argument might not always be useful.
A quick fix can be : ``del X'' to avoid this message.
This message is common in Reddit with 98 occurences.

\newpage
\subsection*{Too restrictive}
\subsubsection*{R0902 : Too many instance attributes}
\pyl{} is often right when it declares this code.
And it is a useful one, because it is a hint of a \tit{bad smell} as a class would be responsible of too many things.
But sometimes, an object really needs to have a direct access to its attributes, and it does not really make sense to group some of those attributes in another object, as they do not have something in common (e.g., the \ttt{ProfileBar} class in pages.py, line 1995).

\medskip
\subsubsection*{W0612: Unused variable X}

This warning can be hard to handle. 
The following code : ``\lstinline|a, b, c = mytuple|'' might generate two of this message if ``\lstinline|a|'' is not used.
But using ``\lstinline|mytuple[1]|'' and ``\lstinline|mytuple[2]|'' to avoid this situation would be worst.
A cleaner solution is to preceed the unused variable by ``\lstinline|dummy_|'' (e.g. ``\lstinline|dummy_a|'').
But this is quite impractical.

\medskip
\subsubsection*{W0223 : Method Y is abstract in class X but not overridden}

It is not mandatory to override an abstract method inherited from an abstract class in Python.
But it is certainly a good practice to do.

\medskip
\subsubsection*{W0142 : Used * or ** magic}

Used when a function or method is called using *args or **kwargs to dispatch arguments. 
This doesn't improve readability but it is useful.
\pyl{} should not try to remove the use of a built-in operator.

\newpage
\subsection*{Useful Checks}

\subsubsection*{R0201 : Method could be a function}

This code has been disabled in the Reddit's configuration but it is really relevant about code quality. 
It is used when there is no reference to the class, suggesting that the method could be used as a static function instead.
These messages are really useful to refactor the code.

\medskip
\subsubsection*{W0201 : Attribute X defined outside \_\_init\_\_}

This bad practice is a common mistake in Reddit.
This could improve code readability if these 144 warnings were fixed.
It is always useful to define all the object members in the \lstinline|__init__| method.

\medskip
\subsubsection*{W0311 : Bad indentation}

Mixing spaces and tabs in a same project is a really bad practice in Python.
If you use tabs only or spaces only, you're fine.
Reddit does mix spaces and tabs in the same file.
This can lead Python to crash.
It is very nice that \pyl{} throw warnings about this practice.


\medskip
\subsubsection*{C0321 : More than one statement on a single line}
This is a useful check because it is not only a bad practice in Python, but it also makes the code harder to read.
When reading the code, it is really easy to miss the instruction.
Furthermore, it can be the cause of bugs, and it makes them harder to find.


\medskip
\subsubsection*{R0912 : Too many branches}
This one is interesting, because it helps to find functions that contain too many \lstinline|if-else|.
It allowed us to find some functions that triggered this code.
It is a useful one because it points out a situation that should be refactored and split in some functions to simplify the code.


\newpage
\section*{\pyl{} usage}
\subsection*{Cost-benefit of using \pyl{}}

\pyl{} can be installed easily via pip (the python package manager).
Since every big python project will have to use pip to maintain dependencies, \pyl{} is really easy to install.
You can run \pyl{} without configuration file or generate one easily (\texttt{pylint --generate-rcfile}).
\pyl{} does integrate with a lot of IDEs (Integrated Development Environment) and CI (Continuous integration) systems.
So, if your project is big, using \pyl{} is easy to run on your project and efforts to deploy are low.
Furthermore, benefits may be important.
However, a little customization could be necessary to avoid unnecessary noise.


\subsection*{Learning cost of using \pyl{}}

Learning cost is nearly inexistent.
You just have to run a command in terminal.
For those that are unfamiliar with command-line, there is also a graphical version.

\subsection*{How long does \pyl{} take to calculate its results}

Results can take a while to output, depending on the size of the project.
Also note that a good \pyl{} configuration file (pylintrc) can drastically decrease the compute time.
Furthermore, it should also helps you to eleminate the noisy messages.

\subsection*{How difficult is it to interpret \pyl{} results}

The interpretation of \pyl{} messages is quite easy.
The real problem is more about their presentation.
The easiest way to understand \pyl{} messages is to have them directly in your IDE or in a CI system.
With that approach, you don't have to look at them by file and line.

\subsection*{How much noise does \pyl{} produce}

\pyl{} can output a lot of noise.
It is higly recommended to use a \pyl{} configuration file (pylintrc) as mentioned before.
You can tweak this file to fit your preferences.
Using a configuration file allows you to have a consistent monitoring.
There are some message codes that can be disabled without having too much consequences on code review.

\section*{Conclusion}

\pyl{} gives good advices to keep a clean and maintenable code.
It is even more useful with a good configuration file (pylintrc).
Furthermore, the analysis is quick and reports can be of different formats.
This tool is well-known and is integrated in some IDEs (Integrated Development Environment) and a lot of CI (Continuous Integration) systems.\\


We would totally recommend this tool for any Python project.
\pyl{} has convinced us, so we decided to use it in our cursus project ``Care4Care'' for the course ``Software Development Project''.
Since we didn't want to add an extra learning curve for our group members, we integrated \pyl{} with Jenkins (a continuous integration server).




\end{document}
