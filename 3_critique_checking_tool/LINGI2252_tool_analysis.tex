% LINGI2252 - Measures & Maintenance
% Code Analysis
\documentclass[11pt, a4paper]{article}
\usepackage[utf8]{inputenc}
\usepackage[UKenglish]{babel}
\usepackage{graphicx}				% Use pdf, png, jpg, or eps§ with pdflatex; use eps in DVI mode
\usepackage{xcolor}
\usepackage{listings}
\usepackage{hyperref}
\usepackage{array}
\usepackage{longtable}
\usepackage{multirow}
\usepackage[babel=true]{csquotes}


\usepackage[T1]{fontenc}

\lstset{%
	basicstyle=\ttfamily\footnotesize,
	commentstyle=\color{green!90!black},
	frame=single,
	keywordstyle=\bfseries\color{blue},
	language=python,
	numberstyle=\color{gray},
%	tabsize=2,
}


\hypersetup{%
	colorlinks=true,
	linkcolor=blue,
	urlcolor=blue
}


\newcommand{\tbf}[1]{\textbf{#1}}
\newcommand{\tit}[1]{\textit{#1}}

\newcommand{\pyl}{\textsf{Pylint}}


\def\blurb{\textsc{Université catholique de Louvain\\
  École polytechnique de Louvain\\
  Pôle d'ingénierie informatique}}
\def\clap#1{\hbox to 0pt{\hss #1\hss}}%
\def\ligne#1{%
  \hbox to \hsize{%
    \vbox{\centering #1}}}%
\def\haut#1#2#3{%
  \hbox to \hsize{%
    \rlap{\vtop{\raggedright #1}}%
    \hss
    \clap{\vbox{\vfill\centering #2\vfill}}%
    \hss
    \llap{\vtop{\raggedleft #3}}}}%
\begin{document}

\begin{titlepage}
\thispagestyle{empty}\vbox to 1\vsize{%
  \vss
  \vbox to 1\vsize{%
    \haut{\raisebox{-2mm}{\includegraphics[width=2.5cm]{logo_epl.jpg}}}{\blurb}{\raisebox{-5mm}{\includegraphics[scale=0.20]{ingi_logo.png}}}
    \vfill
    \ligne{\Huge \textbf{\textsc{LINGI2252}}}
     \vspace{5mm}
    \ligne{\huge \textbf{\textsc{Software Engineering : measures and maintenance}}}
     \vspace{15mm}
    \ligne{\Large \textbf{\textsc{Assignment step 3: \\Providing critiques on the code-checking tool}}}
    \vspace{5mm}
    \ligne{\large{\textsc{November 14, 2014}}}
    \vfill
    \vspace{5mm}
    \ligne{%
         \textsc{Professor\\Kim Mens}
      }
      \vspace{10mm}
    }%
    \ligne{%
         \textsc{Group 23\\Michael Heraly\\Thibault Gerondal}
      }
      \vspace{5mm}
  \vss
  }
\end{titlepage}



\newpage
\section*{Presentation of \pyl{}}

\pyl{} is a free and open-source code-checking tool written in Python.
It has a lot of capabilities : Coding Standard, Error detection, Refactoring help, Fully customizable, Editor integration, IDE integration, UML diagrams and Continuous integration.
In our case, we will only analyse the suggestions that \pyl{} gives, namely : Coding Standard, Error detection and Refactoring help.

\section*{Analysis}
\subsection*{Methodology}

We used our previous analysis of Reddit to have a concrete basis to analyse the messages generated by pylint.
To do so, we have to take into account the fact that Reddit has already a \texttt{pylintrc} file in the project.
This file disables some errors that can be generated by \pyl{}.
We decided to modify their \texttt{pylintrc} file to see how relevant is the tool out of the box.
\medskip


Our first step was to review the disabled messages.
We have reactived them one at a time to see if they had a big impact.
Then, we examined all messages given by \pyl{} and focused on the ones that could be improved.

\subsection*{Disabled messages}
Here are the \pyl{} codes that Reddit disabled for \pyl{}. 
We suspect that these messages generate a lot of noise without real reason.
\\

\begin{longtable}{|l|r|m{9cm}|}
\hline
\textbf{\small{Code}} & \textbf{\small{Diff\footnote{Number of messages added}}} & \textbf{\small{Message}} \\
\hline
\hline
E1103 & 212 & X has no Y member (but some types could not be inferred) \\
\hline
W0212 & 2240  & Access to a protected member of X class \\
\hline
W0223 & 123 & Method Y is abstract in class X but not overridden \\
\hline
C0103 & 33663 & Invalid name ``\%s'' (should match \%s) \\
\hline
C0111 & 3363 & Missing docstring \\
\hline
W0142 & 87 & Used * or ** magic \\
\hline
R0201 & 333 & Method could be a function \\
\hline
R0915 & 31 & Too many statements \\
\hline
I0011 & 0 & Locally disabling \dots (don't need to see things we've explicitly disabled) \\
\hline
\caption{\label{messages} The disabled messages by Reddit can generate a lot of noise. }
\end{longtable}



\newpage
As we can see in Table~\ref{messages}, some enabled codes (e.g. C0111) generate a lot of noise. 
Let's analyse the utility of some of these messages.

\subsection*{Noise}
\subsubsection*{C0111 : Missing docstring}

As we can see, this code generates 3363 messages.
In general, writing a docstring is really useful to help other developers to read the code and understand what the function does.
But sometimes, adding a docstring is not necessary because the function name already gives a good hint of what it does.


\subsubsection*{C0103 : Invalid name ``\%s''  (should match \%s) }

This code generates 33663 messages. 
And we can easily understand why.. This code is about naming convention.
For example, a variable named ``u'' in the code will be reported by \pyl{} with the code C0103 (Invalid variable name ``u'').
But this code does apply also to methods, class, constants, functions, modules and attributes names.

\subsubsection*{W0212 : Access to a protected member of X class}

Used when a protected member (i.e. class member with a name beginning with an underscore) is accessed outside the class or a descendant of the class where it's defined.
This convention is not always respected in Reddit as we can see that \pyl{} gives us 2240 messages.


\subsubsection*{E1103 : X has no Y member (but some types could not be inferred)}

Used when a variable is accessed for a nonexistent member, but astng (Python Abstract Syntax Tree New Generation) was not able to interpret all possible types of this variable.
This error is often triggered but it is nearly always a false-positive since astng was not able to inferred the type of the variable.

\subsection*{Useful Checks}

\subsubsection*{R0201 : Method could be a function}

This code has been disabled in the Reddit's configuration but it is really relevant about code quality. 
It is used when there is no reference to the class, suggesting that the method could be used as a static function instead.
These messages are really useful to refactor the code.




\end{document}
